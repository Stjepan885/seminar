\documentclass{beamer}

\usepackage{circuitikz}
\usepackage{listings}
\usetheme{Warsaw}
\setbeamertemplate{frame numbering}[fraction]
\usecolortheme{crane}
\title{\textbf{IZRADA ELEKTRICNIH DIJAGRAMA}}
\subtitle{LaTeX}
\author{Stjepan Turk \and Martin Luk \and Jan Lamza}


\institute{RiTe
\begin{document}
\begin{frame}
\titlepage
\end{frame}

\begin{frame}[fragile]{korištenje paketa Circuitikz}\vspace{20pt}
\begin{itemize}
\item{Koristimo paket Circuitikz}
\item{ukljucivanje u preambulu}
\item{baziran na tikz}
\item{slicna uporaba kao tikz}
\\
\vspace{30pt}
\begin{centering}
primjer strukture koda :
\end{centering}
\end{itemize}

\begin{lstlisting}
	
		\begin{circuitikz} \draw
		
		code
		;
		\end{circuitikz}
	
\end{lstlisting}
\end{frame}

\begin{frame}[fragile]{Circutikz osnovno}\vspace{20pt}
crtanje u circuitikzu kao u koordinatnom polju :
\begin{lstlisting}
	\begin{circuitikz}\draw
	
		(0,0) -- (0,2) -- (2,2)	
		;
	
	\end{circuitikz}
\end{lstlisting}
\vspace{10pt}
ispisuje :\\
\begin{center}


\begin{circuitikz}\draw
	(0,0) -- (0,2) -- (2,2)	
	;
	\end{circuitikz}
	\end{center}
\end{frame}
\begin{frame}[fragile]{jednostavan elektricni krug}
primjer jednostavnog el. kruga :
\begin{lstlisting}
\begin{circuitikz} \draw
	(0,0) to [battery] (0,4)
	      to [ammeter] (4,4) 
	      -- (4,0)
	      to [lamp] (0,0)				
	;
\end{circuitikz}
\end{lstlisting}

primjer :
\begin{center}
\begin{circuitikz}[scale=0.5] \draw
	(0,0) to [battery] (0,4)
	      to [ammeter] (4,4) 
	      -- (4,0)
	      to [lamp] (0,0)				
	;
\end{circuitikz}
\end{center}
\end{frame}

\begin{frame}{neki poznati elementi}\vspace{10pt}
nazivi za elemente u circuitkzu


\end{frame}

\end{document}
