\documentclass{beamer}

\usepackage{circuitikz}
\usepackage{listings}
\usetheme{Warsaw}
\setbeamertemplate{frame numbering}[fraction]
\usecolortheme{crane}
\title{\textbf{IZRADA ELEKTRICNIH DIJAGRAMA}}
\subtitle{LaTeX}
\author{Stjepan Turk \and Martin Luk \and Jan Lamza}


\institute{RiTeh}
\date{}

\begin{document}
\begin{frame}
\titlepage
\end{frame}

\begin{frame}[fragile]{korištenje paketa Circuitikz}\vspace{20pt}
\begin{itemize}
\item{Koristimo paket Circuitikz}
\item{ukljucivanje u preambulu}
\item{u sebi sadrzi tikz}\\
\vspace{30pt}
\begin{centering}
primjer strukture koda :
\end{centering}
\end{itemize}

\begin{lstlisting}
	
		\begin{circuitikz} \draw
		
		code
		;
		\end{circuitikz}
	
\end{lstlisting}
\end{frame}

\begin{frame}

\begin{circuitikz} \draw
(0,0) to[battery] (0,4)
	  to[ammeter] (4,4)

;
\end{circuitikz}
\end{frame}


\end{document}